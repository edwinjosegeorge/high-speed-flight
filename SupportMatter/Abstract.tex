\chapter*{Abstract}
Quadrotors are agile that can traverse extremely complex environments at high
speeds. This ability had lead to their application in fields such as search and rescue, logistics, security, infrastructure, entertainment, and agriculture.  However, to date, only expert human pilots have been able to fully exploit their capabilities. Autonomous operation with onboard sensing and computation has been limited to low speeds. State-of-the-art methods generally
separate the navigation problem into subtasks: sensing, mapping, and planning. Although this approach has
proven successful at low speeds, the separation it builds upon can be problematic for high-speed navigation in
cluttered environments. The sub-tasks are executed sequentially, leading to increased processing latency and a
compounding of errors through the pipeline.

Here, the paper \autocite{high-speed-flight} propose an end-to-end approach that can autonomously fly quadrotors through complex natural and human-made environments at high speeds, with purely onboard sensing and computation. The key principle is to directly map noisy sensory observations to collision-free trajectories in a receding-horizon \cite{receding_horizon} fashion. This direct mapping drastically reduces processing latency and increases
robustness to noisy and incomplete perception. The sensorimotor mapping is performed by a convolutional network that is trained exclusively in simulation via privileged learning : imitating an expert with access to privileged information. By simulating realistic sensor noise, this approach achieves zero-shot transfer from simulation to
challenging real-world environments that were never experienced during training. This work demonstrates that end-to-end policies trained in
simulation enable high-speed autonomous flight through challenging environments, outperforming traditional
obstacle avoidance pipelines.
