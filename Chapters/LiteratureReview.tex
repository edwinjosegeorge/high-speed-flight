\chapter{Literature Review} 
% Chapter 2 - This tex file refers to the content of your second chapter  
%=======================================================================================

A literature review is a summary of research that has been published about a particular subject. 
It provides the reader with an idea about the current situation in terms of what has 
been done, and what we know. Sometimes it includes suggestions about what needs to be done 
to increase the knowledge and understanding of a particular problem.
\\
The articles used must be from professional journals, which means we can trust that the authors 
are trained professionals, and others have examined their work. Some studies are more easily read 
and summarized than others. Be sure you feel comfortable with your choices, since it is difficult 
to summarize ideas you don't understand.
\\
Once you have found the articles, read them and take notes. Write the literature review from your notes.
The Literature Review is the chapter of the FYP which refers to publications that are related to your 
particular research. The Literature Review
chapter provides a detailed review, discussion and comment on published work that contributes to your own study.
\\
When writing the Literature Review, frequent reference to the work of other 
authors will be made. The two main methods to refer to the work of published sources are:
\begin{itemize}
 \item to use a direct, verbatim (word-for-word) quotation, and
 \item to summarize or paraphrase an autho's work, using your own words.
\end{itemize}

\section{Numbering System}
There is an alternative method for referring to sources which does not use the author's name. 
This alternative method simply allocates a number to the source and the number then refers the
reader to the bibliography at the end. This method is frequently used
in science-based disciplines. If using this system, the reference number is allocated
chronologically, starting with [1] and with the number placed within square brackets. Should a 
reference be made later to a source previously referred to, the earlier number is used.
Note that if the numbering system is used to refer to authors, the bibliography will list 
the references numerically and not alphabetically.

\section{Choice of Verb Tense when Referring to Authors}
When reading academic texts, you may notice that several tenses are used to refer to the work 
of other authors. The tenses often used are simple past, present perfect and simple present. 
Such a range is perfectly acceptable and there are several reasons that influence the choice of tense.

\subsection{Simple Past Tense}
Simple Past Tense tends to be the most frequently used tense to refer to the findings of another 
author's research. 

\subsection{Present Perfect Tense}
It is often used when the focus of the work is on several authors.
\\
e.g. 	
Jolly [2] and Lawrence [3] have studied ...
A number of authors have investigated the strength of … [3, 6, 9]
\\
Present Perfect Tense may also be used when you want to refer to how much or how little 
research has been carried out on a particular topic.
\\
e.g. Very little research has been carried out into the effects of …
\\
Present Tense is often used to refer to generally accepted scientific facts.
\\
e.g. Experimental observations carried out in the past show that … (Smythe, 1995).

\subsection{Model Verbs}
Modal Verbs may be used if you wish to introduce a degree of tentativeness into your comments 
about the work of an author. In this situation the reporting verb will be in the passive voice and
the addition of a modal verb will indicate the degree of confidence attributed to the information.

\section{Range of Verbs to Refer to an Author's Work}
When referring to sources, your writing style will be more effective if you vary the 
choice of verb to refer to the source. The following is a list of frequently used verbs. 
When referring to an author, select a verb that is most appropriate to the context and 
that conveys the author's meaning accurately.
\begin{table}[!ht]
\begin{tabular}{lllll}
Argued & Concluded & Demonstrated & Discussed \\
Examined & Explained & Found & Indicated \\
Investigated & Noted & Pointed out & Presented\\
Proposed & Provided & Reasoned & Recorded \\
Reported & Showed & Stated & Suggested \\
Surveyed & & &  
\end{tabular}
\end{table}

\section{Example}
In \cite{a1} classification and comparison 
of dc-dc converters are described.







