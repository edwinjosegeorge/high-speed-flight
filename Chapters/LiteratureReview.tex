\chapter{Literature Review} 

Various approaches to enable autonomous flight have been proposed in the literature. Some works tackle only perception and build high-quality maps from imperfect measurements, whereas others
focus on planning without considering perception errors. Numerous systems that combine online mapping with traditional planning algorithms have been proposed to achieve autonomous flight in previously unknown environments. A taxonomy of prior works is
presented in Figure \ref{fig:taxonomy}.


\begin{figure}[h]
	\begin{center}
		\includegraphics[width=0.75\textwidth]{taxonomy_navigation.png}
		\caption{Taxonomy of prior works}
		\label{fig:taxonomy}
	\end{center}
\end{figure}

\section{Traditional Models}
Traditional methods mainly involve the division of the navigation task into sensing, mapping and planning sub-tasks. The sensors captures the environment details which are then appropriately mapped to calculate the current state of the quadrotor. This information is used to plan the route for the navigation and associated commands are issued to the rotor. Such as system is attractive from an engineering perspective, because
it enables parallel progress on each component and makes the overall system interpretable. However, it leads to pipelines that largely
neglect interactions between the different stages and thus compound
errors. Their sequential nature also introduces additional latency,
making high-speed and agile maneuvers impossible.
These issues can be mitigated to some degree by careful hand-tuning and engineering via divide-and-conquer principle that has been
prevalent in research on autonomous flight. But such a model imposes fundamental limits on the speed and agility
that a robotic system can achieve.

\section{Recent Models}
Some recent works propose
to learn end-to-end policies directly from data without explicit mapping
and planning stages. These policies are trained by imitating a
human , from experience that was collected in simulation,
or directly in the real world.
Because the number of samples
required to train general navigation policies is very high, existing
approaches impose constraints on the quadrotor’s motion model with reduced maneuverability
and agility. 

More recent work has demonstrated that very agile control
policies can be trained in simulation. Such policies can successfully perform acrobatic maneuvers, but can only
operate in unobstructed free space and are essentially blind to obstacles
in the environment.

\subsection{Reactive Planner}
\subsection{FastPlanner}

\section{Proposed Model}
Here we present an approach to fly a quadrotor at high speeds in
a variety of environments with complex obstacle geometry (Figure \ref{fig:navigation real envt}) while having access to only onboard sensing and computation. By predicting navigation commands directly from sensor
measurements, we decrease the latency between perception and action while simultaneously being robust to perception artifacts, such
as motion blur, missing data, and sensor noise. 

The policy
is exclusively trained in simulation. Stereo matching algorithm is utilized to provide depth images as input
to the policy, showing a strong similarity of the noise models between simulated
and real observations.

The navigation policy is trained via privileged learning on
demonstrations that are provided by a sampling-based expert. The expert
uses Metropolis-Hastings (M-H) sampling to compute a distribution
of collision-free trajectories. This captures the multi-modal nature of
the navigation task where many equally valid solutions can exist. The sampler is bias toward
obstacle-free regions by conditioning it on trajectories from a classic
global planning algorithm.

The neural network policy takes a
noisy depth image and inertial measurements as sensory inputs and
produces a set of short-term trajectories together with an estimate of
individual trajectory costs. The policy
adaptively maps the
predicted trajectories to the best trajectories that have been found by the sampling-based expert. 

\begin{figure}
	\centering
	\begin{subfigure}[b]{\textwidth}
		\centering
		\includegraphics[keepaspectratio=true, width=\textwidth]{Navigation Environment/snow.png}
		\caption{Snowy Environment}
	\end{subfigure}
	\hfill
	\begin{subfigure}[b]{0.48\textwidth}
		\centering
		\includegraphics[keepaspectratio=true, width=\textwidth]{Navigation Environment/forest.png}
		\caption{Dense forest}
	\end{subfigure}
	\hfill	
	\begin{subfigure}[b]{0.48\textwidth}
		\centering
		\includegraphics[keepaspectratio=true, width=\textwidth]{Navigation Environment/narrow gap.png}
		\caption{Narrow gaps}
	\end{subfigure}
	\hfill
	\begin{subfigure}[b]{0.48\textwidth}
		\centering
		\includegraphics[keepaspectratio=true, width=\textwidth]{Navigation Environment/construction site.png}
		\caption{Construction sites}
	\end{subfigure}
	\hfill
	\begin{subfigure}[b]{0.48\textwidth}
		\centering
		\includegraphics[keepaspectratio=true, width=\textwidth]{Navigation Environment/open space.png}
		\caption{Derailed Train}
	\end{subfigure}
	
	\caption{Time-lapse illustrations of agile navigation}
	\label{fig:navigation real envt}
\end{figure}