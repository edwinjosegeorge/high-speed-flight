\chapter{Conclusions and Future Scope}

\section{Conclusion}
Existing autonomous flight systems are highly engineered and modular.
The navigation task is usually split into sensing, mapping, planning,
and control. The separation into multiple modules simplifies the implementation of engineered systems, enables parallel development of each
component, and makes the overall system more interpretable. However,
modularity comes at a high cost: the communication between modules
introduces latency, errors compound across modules, and interactions
between modules are not modeled. In addition, it is an open question if certain subtasks, such as maintaining an explicit map of the
environment, are even necessary for agile flight.

This work replaces the traditional components of sensing, mapping,
and planning with a single function that is represented by a neural
network. This increases the system’s robustness against sensor noise
and reduces the processing latency. It is validated via experiments that this approach
can reach speeds of up to 10 $m/s$ in complex environments and
reduces the failure rate at high speeds by up to 10 times when compared
with the state of the art.
Such a target is achieved by training a neural network to imitate an expert
with privileged information in simulation. 

To cope with the complexity
of the task and to enable seamless transfer from simulation to reality,
several technical contributions are made. These include a sampling based expert, a neural network architecture, and a training procedure,
all of which take the task’s multi-modality into account. The use of an abstract, but sufficiently rich input representation considers real world sensor noise. The combination of these innovations enables the
training of robust navigation policies in simulation that can be directly
transferred to diverse real-world environments without any fine-tuning
on real data.

\section{Future Scope}
Several opportunities can be seen for the future work. 
\begin{enumerate}
	\item Currently, the learned
policy exhibits low success rates at average speeds of $10 m/s$ or
higher in the real world. This is mainly because of the fact that,
at speeds feasible solutions require temporal
consistency over a long time horizon and strong variations of the instantaneous flying speed as a function of the obstacle density. This
requirement makes feasible trajectories extremely sparse in parameter
space, resulting in intractable sampling. Engineering a more complex
expert to tackle this problem can be very challenging and might require
specifically tailored heuristics to find approximate solutions. 
	
	\item The second reason for the performance drop at very
high speeds is the mismatch between the simulated and physical drone
in terms of dynamics and perception. The mismatch in dynamics is
because of aerodynamics effects, motor delays, and dropping battery
voltage.
	
	\item A third reason is
perception latency: Faster sensors can provide more information about
the environment in a smaller amount of time and, therefore, can be
used to provide more frequent updates. This could enable further reducing sensitivity to noise
and promote a quicker understanding of the environment. Perception
latency can possibly be reduced with event cameras
\end{enumerate}