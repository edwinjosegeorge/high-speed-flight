\chapter{Experimental Results}
% Experimental Setup
%==============
The resulting policy can fly a physical quadrotor in natural and
human-made environments at speeds that are unreachable by existing
methods. We achieve this in a zero-shot generalization setting: we train
on randomly generated obstacle courses composed of simple off-the-
shelf objects, such as schematic trees and a small set of convex shapes
such as cylinders and cubes. We then directly deploy the policy in the
physical world without any adaptation or fine-tuning. Our platform
experiences conditions at test time that were never seen during training.
Examples include high dynamic range (when flying from indoor envi-
ronments to outdoor environments), poorly textured surfaces (indoor
environments and snow-covered terrain), thick vegetation in forests,
and the irregular and complex layout of a disaster scenario (Figure 2).
These results suggest that our methodology enables a multitude of
applications that rely on agile autonomous drones with purely onboard
sensing and computation.